\documentclass[compress]{beamer}
%
% Choose how your presentation looks.
%
% For more themes, color themes and font themes, see:
% http://deic.uab.es/~iblanes/beamer_gallery/index_by_theme.html
%
\mode<presentation>
{
	\usetheme{Frankfurt}
	\usecolortheme{default}
	\usefonttheme{default}
	\setbeamertemplate{navigation symbols}{}
	\setbeamertemplate{caption}[numbered]
	\setbeamertemplate{footline}[page number]
} 

\usepackage[english]{babel}
\usepackage[utf8]{inputenc}
\usepackage{amsmath}
\usepackage{amssymb}
\usepackage{mathtools}
\usepackage[T1]{fontenc}
\usepackage{lmodern}
\DeclareMathOperator*{\argmin}{argmin}
\usepackage{multicol}
\usepackage{multirow}
\usepackage{csquotes}
%\usepackage[justification=centering]{caption}
%\usepackage[style=authoryear,backend=biber]{biblatex}
%%% THIS ADDS THE COMMA BETWEEN AUTHOR NAME AND YEAR
%\renewcommand*{\nameyeardelim}{\space}
%\addbibresource{references.bib}
%\usepackage{caption}
%\captionsetup{font=scriptsize,labelfont=scriptsize}
% TiKZ
%\usepackage{tikz}
%\usetikzlibrary{bayesnet}
%\usetikzlibrary{arrows}
\usepackage{svg}

\setbeamerfont{section in head/foot}{size=\tiny}
%\setbeamertemplate{section in head/foot}{}
\setbeamertemplate{subsection in head/foot}{}
%\setbeamerfont{frametitle}{size=\small}

\title[Meaning, Understanding, Quantification]{Meaning, Understanding, and Quantification in the History of Ideas}
%\subtitle{}
\author{\textbf{Jeff Jacobs}\\\small{\texttt{jpj2122@columbia.edu}}}
%\institute{Columbia University\\\texttt{jpj2122@columbia.edu}}
\date{\small{University of Chicago}\\\small{April 28, 2023}}

\setbeamertemplate{mini frames}{}
%
%\makeatletter
%%\setlength{\metropolis@frametitle@padding}{1.6ex}% <- default 2.2 ex
%
%\setbeamertemplate{headline}{%
	%	%\begin{beamercolorbox}[wd=\textwidth, sep=1.5ex]{footline}% <- default 3ex
	%	\begin{beamercolorbox}[wd=\textwidth,sep=3ex]{headline}
		%		\usebeamerfont{page number in head/foot}%
		%		\usebeamertemplate*{frame footer}
		%		\hfill%
		%		\usebeamertemplate*{frame numbering}
		%	\end{beamercolorbox}%
	%}
%\makeatother

\newcommand\Wider[2][3em]{%
	\makebox[\linewidth][c]{%
		\begin{minipage}{\dimexpr\textwidth+#1\relax}
			\raggedright#2
		\end{minipage}%
	}%
}

\begin{document}
	{   
		\setbeamertemplate{section in head/foot shaded}[default][100]
		\setbeamertemplate{mini frame in other subsection}[default][100]
		\frame{ % This is the title slide where I'd like to highlight the sections
			\frametitle{}
			\setbeamertemplate{footline}{} 
			\maketitle  
		}
		\frame{
			\frametitle{Overview}
			\begin{enumerate}
				\item Comparing Translations: Reitter vs. Moore-Aveling and Fowkes
				\begin{itemize}
					\item How did each translation compare to other books of its time?
					\item How does this translation compare to present-day books?
					\item Where does this translation fall relative to the previous two?
				\end{itemize}
				\item Marx Being Marx or Marx Being German?
				\begin{itemize}
					\item How does Marx's language compare with that of his contemporaries?
					\item Did Marx write like a ``Victorian'' economist, or a German sociologist?
				\end{itemize}
				\item Genealogies: \textit{Weltlustig}
			\end{enumerate}
		}
	}
	
	\section{Comparing Translations}
	
	%\section{References}
	%\subsection{References}
	
	%	\begin{frame}[allowframebreaks]{References}
		%		%\bibliographystyle{unsrt}
		%		%\bibliography{references}
		%		\printbibliography
		%	\end{frame}
	%	{   
		%		\setbeamertemplate{section in head/foot shaded}[default][100]
		%		\setbeamertemplate{mini frame in other subsection}[default][100]
		%		\frame{ % This is the title slide where I'd like to highlight the sections
			%			%\frametitle{}
			%			\begin{center}
				%				Thank You!
				%			\end{center}
			%			\setbeamertemplate{footline}{}
			%		}
		%	}
\end{document}
